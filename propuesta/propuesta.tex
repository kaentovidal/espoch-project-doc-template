\chapter{Formulación del trabajo de integración curricular} % poner en mayus

\section{Problema}


\subsection{Antecedentes}
% Detallar investigaciones relacionadas o similares. Por ejemplo si una persona desarrollo una aplicación de citas ponerlo indicando los logros alcanzados, el objetivo con el que se creó y las conclusiones a las que llegó. Revisar estado del arte.
Aquí ponemos cómo hacer \textbf{citas narrativas}... como menciona \textcite{yaqoob2016BDfuture} "la vida es bella", disfruta.

Aquí ponemos cómo hacer \textbf{citas parentéticas}. En la vida nos podemos encontrar con tropiezos pero más allá de eso la vida es bella \parencite{yaqoob2016BDfuture}.
 


\subsection{Formulación del problema}
% Decribir el problema actual
% 

\section{Justificación de la propuesta}
% contrar cómo la propuesta ayudará a resolver este problema

\subsection{Justificación teórica}


\subsection{Justificación aplicativa}

\section{Objetivos}

\subsection{Objetivo general}
\subsection{Objetivos específicos}

\section{Hipótesis y preguntas de investigación} % La hipótesis solo se definirá para proyectos de investigación que involucren operacionalizar variables e indicadores. 


\section{Marco teórico} % 2-4 páginas

\subsection{Conceptos y generalidades}



\subsection{Estado del arte} %Es el alcance de la investigacón hasta la fecha. Es decir, si usted propone el desarrollo de una aplicación de citas, bascar investigaciones actuales  relacionados a su tema, Aquí debe poner, los tipos de algorítmos utilizados para hacer el match (emparejamiento) el diseño del software, pruebas de implementación.

% podríamos poner lo de temario tentativo desglosado



